\documentclass[10pt,a4paper]{article}
\usepackage[utf8]{inputenc}
\usepackage{amsmath}
\usepackage{amsfonts}
\usepackage{amssymb}
\begin{document}

\begin{center}
\textbf{Section 48: Automorphisms of Fields}
\end{center}

\paragraph{Def:} Let $E$ be an algebraic extension of a field $F$. Two elements $\alpha,\beta \in E$ are \textit{conjugate} over $F$ if $irr(\alpha, F) = irr(\beta, F)$, that is, if $\alpha$ and $\beta$ are zeros of the same irreducible polynomial over $F$.

\paragraph{Example:} Conjugate complex numbers $a+bi$ and $a-bi$ are roots of the same polynomial.

\paragraph{Conjugation Isomorphisms:} Let $F$ be a field, and let $\alpha$ and $\beta$ be algebraic over $F$ with $deg(\alpha, F) = n$. The map $\psi_{\alpha,\beta}: F(\alpha) \to F(\beta)$ defined by
$$ \psi_{\alpha, \beta}(c_0+c_1\alpha+\dots+c_{n-1}\alpha^{n-1}) = c_0 + c_1\beta+\dots+c_{n-1}\beta^{n-1}$$
for $c_i \in F$ is an isomorphism of $F(\alpha)$ onto $F(\beta)$ if and only if $\alpha$ and $\beta$ are conjugate over $F$.

\paragraph{Corollary:} Let $\alpha$ be algebraic over a field $F$. Every isomorphism $\psi$ mapping $F(\alpha)$ onto a subfield of $\bar{F}$ such that $\psi(a) = a$ for $a \in F$ maps $\alpha$ onto a conjugate $\beta$ of $\alpha$ over $F$. Conversely, for each conjugate $\beta$ of $\alpha$ over $F$, there exists exactly one isomorphism $\psi_{\alpha, \beta}$ of $F(\alpha)$ onto a subfield of $\bar{F}$ mapping $\alpha$ onto $\beta$ and mapping each $a \in F$ onto itself.

\paragraph{Corollary:} Let $f(x) \in \mathbb{R}[x]$. If $f(a+bi)=0$ for $a+bi \in \mathbb{C}$, where $a,b \in \mathbb{R}$, then $f(a-bi) = 0$ also. 

\paragraph{Def:} An isomorphism of a field onto itself is an \textit{automorphism} of the field.

\paragraph{Def:} If $\sigma$ is an isomorphism of a field $E$ onto some field, then an element $a$ of $E$ is \textit{left fixed} by $\sigma$ if $\sigma(a) = a$. A collection $S$ of isomorphisms of $E$ \textit{leaves} a subfield $F$ of $E$ \textit{fixed} if each $a \in F$ is left fixed by every $\sigma \in S$. If $\{\sigma\}$ leaves $F$ fixed, then $\sigma$ leaves $F$ fixed.

\paragraph{Thm.} Let $\{ \sigma_i | i \in I \}$ be a collection of automorphisms of a field $E$. Then the set $E_{\{\sigma_i\}}$ of all $a \in E$ left fixed by every $\sigma_i$ for $i \in I$ forms a subfield of $E$.

\paragraph{Def:} The field $E_{\{\sigma_i\}}$ is the \textit{fixed field} of $\{\sigma_i | i \in I \}$. For a single automorphism $\sigma$, we shall refer to $E_{\{\sigma\}}$ as the \textit{fixed subfield} of $\sigma$.

\paragraph{Thm.} The set of all automorphisms of a field $E$ is a group under function composition.

\paragraph{Note:} These automorphisms are basically permutation groups of the field.

\paragraph{Thm.} Let $E$ be a field and let $F$ be a subfield of $E$. Then the set $G(E/F)$ of all automorphisms of $E$ leaving $F$ fixed forms a subgroup of the group of all automorphisms of $E$. Furthermore, $F \leq E_{G(E/F)}$.

\paragraph{Def:} The group $G(E/F)$ of the preceding theorem is the \textit{group of automorphisms} of $E$ leaving $F$ fixed, or, the \textit{group of} $E$ \textit{over} $F$.

\paragraph{Note:} The notation $G(E/F)$ is a little misleading since it is not useful to think of this group as a quotient space. Instead think of it as referring to that $E$ is an extension field of $F$.

\paragraph{Thm:}  Let $F$ be a finite field of characteristic $p$. Then the map $\sigma_p:F \to F$ via $\sigma_p(a) = a^p$ for $a \in F$ is an automorphism called the \textit{Frobenius automorphism} of $F$. Also, $F_{\{\sigma_p\}} \simeq \mathbb{Z}_p$.

\paragraph{Note:} The Frobenius automorphism is important because it is the generator of the group of automorphisms on a field.

\paragraph{Selected Exercises:}

\paragraph{39a.} Prove that and automorphism  of a field $E$ carries elements that are squares of elements in $E$ onto elements that are squares that are elements of $E$.

\paragraph{39b.} Prove that an automorphism of the field $\mathbb{R}$ carries positive numbers onto positive numbers.

\paragraph{39c.} Prove that if $\sigma$ is an automorphism of $\mathbb{R}$ and $a <b$, where $a,b \in \mathbb{R}$, then $\sigma(a) < \sigma(b)$.

\paragraph{39d.} Finally, prove that the only automorphism of $\mathbb{R}$ is the identity automorphism.

\end{document}