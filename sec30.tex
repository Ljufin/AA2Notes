\documentclass[10pt,a4paper]{article}
\usepackage[utf8]{inputenc}
\usepackage{amsmath}
\usepackage{amsfonts}
\usepackage{amssymb}
\begin{document}

\begin{center}
\textbf{Section 30: Vector Spaces}
\end{center}

\paragraph{Def:} Let $F$ be a field. A \textit{vector space} over $F$ consists of an abelian group $V$ under addition together with an operation of scalar multiplication of each element of $V$ by each element of $F$ on the left, such that for all $a,b \in F$ and $\alpha, \beta \in V$, the following conditions are satisfied:
\begin{enumerate}
\item $a\alpha \in V$
\item $a(b\alpha) = (ab)\alpha$
\item $(a+b)\alpha = (a\alpha)+(b\alpha)$
\item $a(\alpha + \beta) = (a\alpha) + (a\beta)$
\item $1 \alpha = \alpha$
\end{enumerate}

The elements of $V$ are \textit{vectors} and the elements of $F$ are scalars.

\paragraph{Note:} Let $E$ be an extension field over a field $F$. Then $E$ is vector space over $F$ with the usual addition in $E$ and scalar multiplication is the usual multiplication $\alpha a \in E$ where $\alpha \in E$ and $a \in F$.

\paragraph{Thm.} If $V$ is a vector space over $F$, then $0\alpha = 0, a0 =0$ and $(-a)\alpha=a(-\alpha) = -(a\alpha)$ for all $ a \in F$ and $\alpha \in V$.

\paragraph{Def:} Let $V$ be a vector space over $F$. The vectors in the subset $S = \{ \alpha_i | i\in I\}$ of $V$ \textit{span} $V$ if for every $\beta \in V$, we have 
$$ \beta = a_1\alpha_{i_1}+a_2\alpha_{i_2}+\dots + a_n\alpha_{i_n}$$
for some $a_j \in F$ and $\alpha_{i_j} \in S, j= 1, \dots, n$. A vector $\sum_{j=1}^n a_j\alpha_{i_j}$ is a \textit{linear combination} of the $\alpha_{i_j}$.

\paragraph{Example:}  Let $F$ be a field and $E$ an extension field of $F$. Let $\alpha \in E$ be algebraic over $F$. Then $F(\alpha)$ is a vector space over $F$ and is spanned by vectors of of the form $a_0+a_1\alpha+ \dots+ a_{n-1}\alpha^{n-1}$, where $n=deg(\alpha, F)$.

\paragraph{Def:} A vector space $V$ over a field $F$ is \textit{finite dimensional} if there is a finite subset of $V$ whose vectors span $V$.

\paragraph{Example:} If $F \leq E$ and $\alpha \in E$ is algebraic over $F$, $F(\alpha)$ is a finite dimensional vector space over $F$. This happens because the degree of the irreducible polynomial of $\alpha$ is finite and therefore the spanning vector will have that number of components.

\paragraph{Def:} The vectors in the subset $S = \{ \alpha_i | i\in I\}$ of a vector space $V$ over a field $F$ are \textit{linearly independent} over $F$ if, for any distinct vectors $\alpha_{i_j} \in S$, coefficients $a_j \in F$ and $n \in \mathbb{Z}^+$, we have $\sum_{j=1}^n a_j \alpha_{i_j} = 0$ in $V$ only if $a_j =0$ for $j= 1,\dots,n$. If the vectors are not linearly independent, they are \textit{linearly dependent}.

\paragraph{Note:} Another way to think of linear independence is that there is only way to represent the 0 vector. Also, if the vectors are linearly dependent, one will be a linear combination of the remaining vectors.

\paragraph{Example:} Back to the $F(\alpha)$ case in previous examples, the vectors look like $v =a_0+a_1\alpha, \dots, a_{n-1}\alpha^{n-1}$. Setting $v=0$,
$$0 = a_0+a_1\alpha+ \dots+ a_{n-1}\alpha^{n-1} = 0+0\alpha+ \dots+ 0\alpha^{n-1}$$
Therefore, the $\alpha$'s are linearly independent.

\paragraph{Def:} If $V$ is a vector space over a field $F$, the vectors in a subset $B = \{ \beta_i | i \in I \}$ of $V$ form a \textit{basis} for $V$ over $F$ if they span $V$ and are linearly independent.

\paragraph{Lemma:} Let $V$ be a vector space over a field $F$, and let $\alpha \in V$. If $\alpha$ is a linear combination of vectors $\beta_i$ in $V$ for $i=1,\dots,m$ and each $\beta_i$ is a linear combination of vectors $\gamma_j$ in $V$ for $j=1,\dots,n$, then $\alpha$ is a linear combination of the $\gamma_i$.

\paragraph{Thm.} In a finite dimensional vector space, every finite set of vectors spanning the space contains a subset that is a basis.

\paragraph{Note:} This theorem means that we can always generate a finite dimensional vector space from a finite number of basis vectors. (whoops, got ahead of the text)

\paragraph{Corollary:} A finite dimensional vector space has a finite basis.

\paragraph{Thm.} Let $S = \{\alpha_1,\dots,\alpha_r\}$ be a finite set of linearly independent vectors of a finite dimensional vector space $V$ over a field $F$. Then $S$ can be enlarged to a basis for $V$ over $F$. Furthermore, if $B = \{\beta_1, \dots, \beta_n\}$ is any basis for $V$ over $F$, then $r \leq n$.

\paragraph{Corollary:} Any bases of a finite dimensional vector space $V$ over $F$ have the same number of elements.

\paragraph{Def:} If $V$ is a finite dimensional vector space over a field $F$, the number of elements in a basis is the \textit{dimension} of $V$ over $F$.

\paragraph{Example:} Let $E$ be an extension field of $F$ and let $\alpha \in E$ be algebraic over $F$. If $deg(\alpha, F) = n$, then the dimension of $F(\alpha)$ is $n$.

\paragraph{Thm.} Let $E$ be an extension field of $F$, and let $\alpha \in E$ be algebraic over $F$. If $deg(\alpha, F) =n$, then $F(\alpha)$ is an $n$-dimensional vector space over $F$ with basis $\{1, \alpha, \dots, \alpha^{n-1} \}$. Furthermore, every element $\beta \in F(\alpha)$ is algebraic over $F$, and $deg(\beta, F) \leq deg(\alpha, F)$. 

\paragraph{Selected Exercises:}

\paragraph{23.} Prove that every finite dimensional vector space $V$ of dimension $n$ over a field $F$ is isomorphic to the vector space $F^n$.

\paragraph{24a.} Let $\{ \beta_i | i \in I\}$ be a basis for $V$ over $F$, show that a linear transformation $\phi: V \to V'$ is completely determined by the vectors $\phi(\beta_i) \in V'$.

\paragraph{26.} Let $V$ be a vector space over a field $F$, and let $S$ be a subspace of $V$. Define the \textit{quotient space} $V/S$, and show that it is a vector space over $F$.



\end{document}