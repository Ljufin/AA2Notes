\documentclass[10pt,a4paper]{article}
\usepackage[utf8]{inputenc}
\usepackage{amsmath}
\usepackage{amsfonts}
\usepackage{amssymb}
\begin{document}

\begin{center}
\textbf{Section 27: Prime and Maximal Ideals}
\end{center}

\paragraph{Factoids:} This section explores the connection that factor rings have to integral domains and to fields. Here are some interesting factoids:
\begin{itemize}
\item A factor ring of an integral domain may be a field. Example: $\mathbb{Z}/p\mathbb{Z} \simeq \mathbb{Z}$.
\item A factor ring of a ring may be an integral domain even though the original ring is not. Example: $(\mathbb{Z} \times \mathbb{Z})/N \simeq \mathbb{Z}$ where $N = \{(0,n)|n \in \mathbb{Z}\}$.
\item If $R$ is not even an integral domain, it is still possible for $R/N$ to be a field. Example: $\mathbb{Z}_6/\{0,3\} \simeq \mathbb{Z}_3$.
\item A factor ring may also have a worse structure than the original ring. Example: $\mathbb{Z}$ is an integral domain, but $\mathbb{Z}/6\mathbb{Z} \simeq \mathbb{Z}_6$ is not.
\end{itemize}

\paragraph{Thm.} If $R$ is a ring with unity, and $N$ is an ideal of $R$ containing a unit, then $N=R$.

\paragraph{Corollary:} A field contains no proper nontrivial ideals.

\paragraph{Note:} This makes the factor rings of a field not very interesting. The factor ring will either be $\{0\}$ or the field itself.

\paragraph{Def:} The \textit{maximal ideal} of a ring $R$ is an ideal $M$ different from $R$ such that there is no proper ideal $N$ of $R$ properly containing $M$.

\paragraph{Thm.} Let $R$ be a commutative ring with unity. Then $M$ is a maximal ideal of $R$ if and only if $R/M$ is a field.

\paragraph{Proof Sketch:} Suppose that $M$ is a maximal ideal in $R$ and that there is an element in $R/M$ that does not have a multiplicative inverse. We can then construct an ideal of $R$ that contains $M$, contradicting the original assumption. Therefore, every element in $R/M$ needs to have a multiplicative inverse and is thus a field.

\paragraph{Example:} $p\mathbb{Z}$ is a maximal ideal of $\mathbb{Z}$. Therefore $\mathbb{Z}/p\mathbb{Z} \simeq \mathbb{Z}_p$ is a field.

\paragraph{Corollary:} A commutative ring with unity is a field if and only if it has no proper nontrivial ideals.

\paragraph{Def:} An ideal $N \neq R$ in a commutative ring $R$ is a \textit{prime ideal} if $ab \in N$ implies that either $a \in N$ or $b \in N$ for $a,b \in R$.

\paragraph{Example:} $\{0\}$ is a prime ideal in any integral domain.

\paragraph{Note:} Prime ideals are based on considering the zero divisors of factor rings. The definition is derived from the fact that $$ (a+N)(b+N)+N \implies a+N=N \text{ or } b+N=N$$ if we want our factor ring to be an integral domain. This is stated in  the following theorem.

\paragraph{Thm.} Let $R$ be a commutative ring with unity, and let $N \neq R$ be an ideal in $R$. Then $R/N$ is an integral domain if and only if $N$ is a prime ideal in $R$.

\paragraph{Corollary:} Every maximal ideal in a commutative ring with unity is a prime ideal.

\paragraph{Summary:} Maximal and prime ideals are a very important concept to understand going forward. Here is a summary of the major results so far:
\begin{enumerate}
\item An ideal $M$ of $R$ is maximal iff $R/M$ is a field.
\item An ideal $N$ of $R$ is prime iff $R/N$ is an integral domain.
\item Every maximal ideal of $R$ is a prime ideal.
\end{enumerate}
Notice that these 3 statements form a hierarchy of ideals. Just like how a field is an integral domain with more requirements, a maximal ideal is a prime ideal with more requirements.

\paragraph{Goal:} To show that the rings $\mathbb{Z}$ and $\mathbb{Z}_n$ form foundations upon which all rings with unity rest, and that $\mathbb{Q}$ and $\mathbb{Z}_p$ perform a similar service for all fields.

\paragraph{Thm.} If $R$ is a ring with unity 1, then the par $\phi: \mathbb{Z} \to R$ given by $\phi(n) = n \cdot 1$ for $n \in \mathbb{Z}$ is a homomorphism of $\mathbb{Z}$ into $R$.

\paragraph{Corollary:} If $R$ is a ring with unity and characteristic $n >1$, then $R$ contains a subring isomorphic to $\mathbb{Z}_n$. If $R$ has characteristic 0, then $R$ contains a subring isomorphic to $\mathbb{Z}$.

\paragraph{Snarky Remark:} It seems that the major results in this chapter are all in the corollaries. It seems that the author should label his theorems as lemmas and his corollaries as theorems!

\paragraph{Proof Sketch:} Make use of the last theorem and "throw" the integers into a ring with characteristic $n$ and then with characteristic 0. 

\paragraph{Thm.} A field $F$ is either of prime characteristic $p$ and contains a subfield isomorphic to $\mathbb{Z}_p$. If $R$ has characteristic 0, then $R$ contains a subring isomorphic to $\mathbb{Q}$.

\paragraph{Def:} The fields $\mathbb{Z}_p$ and $\mathbb{Q}$ are \textit{prime fields}.

\paragraph{Def:} If $R$ is a commutative ring with unity and $a \in R$, the ideal $\{ ra | r \in R\}$ of all multiples of $a$ is the \textit{principal ideal generated by} $a$ and is denoted by $\langle a \rangle$. An ideal $N$ of $R$ is a \textit{principal ideal} if $N = \langle a \rangle$ for some $a \in R$.

\paragraph{Example:} Every every ideal of $\mathbb{Z}$ is of the form $n\mathbb{Z}$, which is generated by $n$, so every ideal of $\mathbb{Z}$ is a principal ideal.

\paragraph{Thm.} If $F$ is a field, every ideal in $F[x]$ is principal.

\paragraph{Thm.} An ideal $\langle p(x) \rangle \neq \{0\}$ of $F[x]$ is maximal if and only if $p(x)$ is irreducible over $F$.

\paragraph{Example:} $x^2-2$ is irreducible in $\mathbb{Q}[x]$. Therefore, $\mathbb{Q}[x]/\langle x^2 -2 \rangle$ is a field.

\paragraph{Thm.} Let $p(x)$ be an irreducible polynomial in $F[x]$. If $p(x)$ divides $r(x)s(x)$ for $r(x),s(x) \in F[x]$, then either $p(x)$ divides $r(x)$ or $p(x)$ divides $s(x)$.

\begin{flushleft}
\textit{Proof:} Let $p(x)$ divide $r(x)s(x)$.\\
Then $r(x)s(x)$ must be a multiple of $p(x)$ and therefore $r(x)s(x) \in \langle p(x) \rangle$.\\

\end{flushleft} 


\end{document}