\documentclass[10pt,a4paper]{article}
\usepackage[utf8]{inputenc}
\usepackage{amsmath}
\usepackage{amsfonts}
\usepackage{amssymb}
\begin{document}

\begin{center}
\textbf{Section 35: Series of Groups}
\end{center}

\paragraph{Def:} A \textit{subnormal series} of a group $G$ is a finite sequence $H_0, H_1, \dots, H_n$ of subgroups of $G$ such that $H_i < H_{i+1}$ and $H_i$ is a normal subgroup of $H_{i+1}$ with $H_0 = \{e\}$ and $H_n = G$. A \textit{normal series} of $G$ is a finite sequence $H_0, H_1, \dots, H_n$ of normal subgroups of $G$ such that $H_i < H_{i+1}, H_0 = \{e\}$, and $H_n=G$.

\paragraph{Note:} The difference between a subnormal series and a normal series is that in a subnormal series the groups just have to be normal subgroups of the next group in the series, while in the normal series, every subgroup has to be normal under the original group.

\paragraph{Note:} A normal series is always subnormal, but a subnormal series is not always normal.

\paragraph{Def:} A subnormal series $\{ K_j \}$ is a \textit{refinement} of a subnormal series $\{H_j\}$ of a group $G$ if $\{H_i\} \subseteq \{K_j\}$, that is, if each $H_i$ is one of the $K_j$.

\paragraph{Def:} Two subnormal series $\{ H_j \}$ and $\{K_j\}$ of the same group $G$ are \textit{isomorphic} if there is a 1-1 correspondence between the collections of factor groups $\{ H_{i+1}/H_i\}$ and $\{K_{j+1}/K_j\}$ such that the corresponding factor groups are isomorphic.

\paragraph{Note:} The way we can compare two series is to compare the individual steps by looking at factor groups. Two subnormal series are isomorphic if the steps they take are the same.

\paragraph{Zassenhaus Lemma:} Let $H$ and $K$ be subgroups of a group $G$ and let $H^*$ and $K^*$ be normal subgroups of $H$ and $K$ respectively. Then
\begin{enumerate}
\item $H^*(H \cap K^*)$ is a normal subgroup of $H^*(H \cap K)$.
\item $K^*(H^* \cap K)$ is a normal subgroup of $K^*(H \cap K)$.
\item $H^*(H \cap K)/H^*(H \cap K^*) \simeq K^*(H \cap K)/K^*(H^* \cap K) \simeq (H \cap K)/\left[ (H^* \cap K)(H \cap K^*) \right]$
\end{enumerate}

\paragraph{Schreier Theorem:} Two subnormal series of a group $G$ have isomorphic refinements.

\paragraph{Note:} I have completely lost track of what we are supposed to be doing in this section. I understand the results, but the point of it all is lost on me.

\paragraph{Def:} A subnormal series $\{H_i\}$ of a group $G$ is a \textit{composition series} if all the factor groups $H_{i+1}/H_i$ are simple. A normal series $\{H_i\}$ of $G$ is a \textit{principal series} if all the factor groups $H_{i+1}/H$ are simple.

\paragraph{Jorden-Holder Theorem:} Any two composition or principle series of a group $G$ are isomorphic.

\paragraph{Note:} This theorem says that there is only really one way to make a composition series of a group. For finite groups, we divide up the group into simple factor groups. This factorization is unique, similar to how every natural number has a unique prime factorization.

\paragraph{Thm.} If $G$ has a composition series, and if $N$ is a proper normal subgroup of $G$, then there exists a composition series containing $N$.

\paragraph{Def:} A group $G$ is \textit{solvable} if it has a composition series $\{H_i\}$ such that all factor groups $H_{i+1}/H_i$ are abelian.

\paragraph{Example:} The group $S_5$ is not solvable since
$$ \{e\} < A_5 < S_5$$
is a composition series. Since $A_5$ is not abelian, $S_5$ is not solvable.

\paragraph{Def:} The series
$$ \{e\} \leq Z(G) \leq Z_1(G) \leq Z_2(G) \leq \dots$$
where $Z_i$ represents the center of a factor factor group of $G$ is the \textit{ascending central series} of the group $G$.

\paragraph{22.} Let $H^*$, $H$, and $K$ be subgroups of $G$ with $H^*$ normal in $H$. Show that $H^* \cap K$ is normal in $H \cap K$.

\paragraph{25.} Show that a finite direct product of solvable groups is solvable.


\end{document}