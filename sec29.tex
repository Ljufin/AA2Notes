\documentclass[10pt,a4paper]{article}
\usepackage[utf8]{inputenc}
\usepackage{amsmath}
\usepackage{amsfonts}
\usepackage{amssymb}
\begin{document}

\begin{center}
\textbf{Section 29: Introduction to Extension Fields}
\end{center}

\paragraph{Goal:} The results in this section will allow us to prove that every nonconstant polynomial has a zero.

\paragraph{Def:} A field $E$ is an \textit{extension field} of a field $F$ if $F \leq E$.

\paragraph{Kronecker's Theorem:} Let $F$ be a field and let $f(x)$ be a nonconstant polynomial in $F[x]$. Then there exists an extension field $E$ of $F$ and an $\alpha \in E$ such that $f(\alpha) =0$.

\begin{flushleft}
\textit{Proof:} \\
$f(x)$ has a factorization in $F[x]$ into polynomials that are irreducible over $F$. 
Let $p(x)$ be an irreducible polynomial in such a factorization.
It is E.T.S that there exists an extension field of $F$ that contains a root of $p(x)$.
We can get our required extension field by noting that, since $\langle p(x) \rangle$ is a maximal ideal in $F[x]$, $F[x]/\langle p(x) \rangle$ must be a field. To show that $F$ is a subfield of $E = F[x]/ \langle  p(x) \rangle$, we construct the following map, $\psi: F \to E$ via $$ \psi(a) = a + \langle p(x) \rangle$$ for $a \in F$. $\psi$ is 1-1 since $\psi(a) = \psi(b)$ implies that $a-b \in \langle p(x) \rangle$ and since $a,b \in F$, $a-b=0$ and $a=b$.  $\psi$ is a homomorphism that maps $F$ 1-1 onto a subfield of $F[x]/\langle p(x) \rangle$. the existence of this subfield shows that $E$ is an extension field of $F$. 

Now we need to show that $E$ contains a zero of $p(x)$. Let $\alpha = x + \langle p(x) \rangle \in E$. We can use the evaluation homomorphism $\phi_a:F[x] \to E$ at $\alpha$, plugging into the polynomial $p(x)$. 
$$ \phi_\alpha(p(x)) = a_0 + a_1(x+\langle p(x) \rangle)+ \dots + a_n(x+\langle p(x) \rangle)^n$$ Since $x$ is a representative of the coset $\alpha = x + \langle p(x) \rangle$, we can rewrite our evaluation as 
$$\phi_\alpha(p(x)) = p(\alpha) = (a_0+a_1x+\dots+a_nx^n) + \langle p(x) \rangle$$
$$ = p(x) + \langle p(x) \rangle = \langle p(x) \rangle = 0$$
Therefore, we have found an element $\alpha$ in an extension field $E$ of $F$ such that $p(\alpha) = 0$. $\blacksquare$

\paragraph{Proof Synopsis:} The proof of Kronecker's Theorem comes in two parts. The first, and more challenging part, involves constructing an extension field based off that fact that our polynomial is irreducible in the original field. In the second part, we chose an element in our extension field such that our polynomial evaluates to the ideal generated by $p(x)$, which is the zero in the extension field, but also in our original field.
\end{flushleft}

\paragraph{Classical Example:} Take $F = \mathbb{R}$ and $f(x) = x^2+1$. $\mathbb{R}$ is a subfield of $\mathbb{R}[x]/\langle x^2+1 \rangle$. Now choose an $\alpha$, 
$$ \alpha = x + \langle x^2 +1 \rangle$$
We can now plug this into $f(x)$, keeping in mind that we are working in the field $\mathbb{R}[x]/\langle x^2 + 1 \rangle$.
$$ f(\alpha) = \alpha^2 + 1 = (x + \langle x^2 +1 \rangle)^2 +(1 + \langle x^2 +1 \rangle)$$
$$ = (x^2+1) + \langle x^2 + 1 \rangle = \langle x^2 + 1 \rangle = 0$$
You may recognize the field $\mathbb{R}[x]/\langle x^2+1 \rangle$ as $\mathbb{C}$.



\end{document}