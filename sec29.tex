\documentclass[10pt,a4paper]{article}
\usepackage[utf8]{inputenc}
\usepackage{amsmath}
\usepackage{amsfonts}
\usepackage{amssymb}
\begin{document}

\begin{center}
\textbf{Section 29: Introduction to Extension Fields}
\end{center}

\paragraph{Goal:} The results in this section will allow us to prove that every nonconstant polynomial has a zero.

\paragraph{Def:} A field $E$ is an \textit{extension field} of a field $F$ if $F \leq E$.

\paragraph{Kronecker's Theorem:} Let $F$ be a field and let $f(x)$ be a nonconstant polynomial in $F[x]$. Then there exists an extension field $E$ of $F$ and an $\alpha \in E$ such that $f(\alpha) =0$.

\begin{flushleft}
\textit{Proof:} \\
$f(x)$ has a factorization in $F[x]$ into polynomials that are irreducible over $F$. 
Let $p(x)$ be an irreducible polynomial in such a factorization.
It is E.T.S that there exists an extension field of $F$ that contains a root of $p(x)$.
We can get our required extension field by noting that, since $\langle p(x) \rangle$ is a maximal ideal in $F[x]$, $F[x]/\langle p(x) \rangle$ must be a field. To show that $F$ is a subfield of $E = F[x]/ \langle  p(x) \rangle$, we construct the following map, $\psi: F \to E$ via $$ \psi(a) = a + \langle p(x) \rangle$$ for $a \in F$. $\psi$ is 1-1 since $\psi(a) = \psi(b)$ implies that $a-b \in \langle p(x) \rangle$ and since $a,b \in F$, $a-b=0$ and $a=b$. 
\end{flushleft}

\end{document}