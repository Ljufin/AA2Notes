\documentclass[10pt,a4paper]{article}
\usepackage[utf8]{inputenc}
\usepackage{amsmath}
\usepackage{amsfonts}
\usepackage{amssymb}
\begin{document}

\begin{center}
Section 26: Homomorphisms and Factor Rings
\end{center}

\paragraph{Def:} A map $\phi$ of a ring $R$ into a ring $R'$ is a \textit{homomorphism} if $$ \phi(a+b) = \phi(a)+\phi(b)$$ and $$\phi(ab)=\phi(a)\phi(b)$$ for all elements $a,b \in R$.

\paragraph{Example:} \textit{The Projection Homomorphisms:} Let $R_1, R_2, \dots, R_n$ be rings. For each $i$, the map $\pi_i:R_1 \times R_2 \times \dots \times R_n \to R_i$ defined by $\pi_i(r_1,r_2,\dots,r_n) = r_i$ is a homomorphism. This is called the "projection onto the $i$th component of $R_i$." It is very easy to see that this meets all the requirements of the homomorphism.

\paragraph{}A number of key results easily transfer over from group homomorphisms.

\paragraph{Thm:} Let $\phi$ be a homomorphism of a ring $R$ into a ring $R'$. The following properties hold:
\begin{enumerate}
\item If 0 is the additive identity in $R$, then $\phi(0) = 0'$ is the additive identity in $R'$.\\
\item If $a\in R$, then $\phi(-a) = -\phi(a)$.\\
\item If $S$ is a subring of $R$, then $\phi(S)$ is a subring of $R'$.\\
\item If $S'$ is a subring of $R'$, then $\phi^{-1}(S')$ is a subring of $R$.\\
\item If $R$ has unity 1, then $\phi(1)$ is unity for $\phi(R)$.
\end{enumerate}

\paragraph{Def:} Let a map $\phi: R \to R'$ be a ring homomorphism. The subring $$\phi^{-1}(0') = \{r \in R | \phi(r) = 0'\} $$ is the \textit{kernel} of $\phi$, denoted $Ker(\phi)$.

\paragraph{Factor(Quotient) Rings:} Now that we have transferred what we know about group homomorphisms to ring homomorphisms, we can extend the concept of a factor group to get a factor ring. 

\paragraph{Thm:} Let $\phi:R \to R'$ be a ring homomorphism with kernel $H$. Then the additive cosets of $H$ form a ring $R/H$ whose binary operations are defined by $$ (a+H)+(b+H) = (a+b)+Hb$$, and $$(a+H)(b+H) = (ab)+H$$.

\paragraph{Note:} The map $\mu: R/H \to \phi(R)$ defined by $\mu(a+H) = \phi(a)$ is an isomorphism.

\paragraph{Example:} This theorem allows us to prove that $\mathbb{Z}/n\mathbb{Z}$ is isomorphic to $\mathbb{Z}_n$.

\paragraph{Thm:} Let $H$ be a subring of $R$. The multiplication of additive cosets of $H$ is well-defined if and only if $ah \in H$ and $hb \in h$ for all $a,b \in R$ and $h \in H$.

\paragraph{Def:} An adative subgroup $N$ of a ring $R$ satisfying the properties:
\begin{itemize}
\item $aN \subseteq N$\\
\item $Nb \subseteq N$
\end{itemize}
for all $a,b \in R$ is an \textit{ideal}.

\paragraph{Note:} The book's notation is very confusing here. The requirements are easier to understand if they are put this way: $$\forall x \in N, \forall r \in R: \quad xr \in N, rx \in N$$(Source: Wikipedia).

\paragraph{Note:} Ideals are the ring equivalent of normal subgroups.

\paragraph{Note:} If $R$ is a field, then the only ideals are $\{0\}$ and $R$ itself. 0 will obviously obliterate anything that gets multiplied by it, and the entire field will contain everything. It is very easy to escape closure if you pick a smaller subring of $R$.

\paragraph{Example:} $n\mathbb{Z}$ is an ideal in $\mathbb{Z}$. $n\mathbb{Z}$ is a subring and $s(mn) = (nm)s = n(ms) \in n\mathbb{Z}$ for all $s \in \mathbb{Z}$. What we did was to show that both left and right cosets are a subset of $n\mathbb{Z}$

\paragraph{Def:} Let $N$ be an ideal of a ring $R$. The additive cosets of $N$ form a ring $R/N$ with the binary operations by defined as $$ (a+N)+(b+N) = (a+b)+N$$ and $$(a+N)(b+N) = ab+N$$. This ring is the \textit{factor ring} of \textit{quotient ring} of $R$ by $N$.

\paragraph{Thm:} Let $N$ be and ideal of a ring $R$. Then $\gamma : R \to R/N$ given by $\gamma (y) = x+N$ is a ring homomorphism with $Ker(\gamma) = N$.

\paragraph{Fundamental Homomorphism Theorem:} Let $\phi:R \to R'$ be a ring homomorphism with kernel $N$. Then $\phi(R)$ is a ring, and the map $\mu: R/N \to \phi(R)$ given by $\mu(x+N)=\phi(x)$ is an isomorphism. If $\gamma:R \to R/N$ is the homomorphism given by $\gamma(x) = x+N$, then for each $x \in R$, we have $\phi(x) = \mu(\gamma(x))$.

\paragraph{Note:} This theorem is basically saying that we can consider a homomorphism in two pieces. The ring is first mapped to a factor ring and then another mapping goes from the factor ring to $\phi(R)$. The book says it this way, \textit{"every ring homomorphism with domain R gives rise to a factor ring R/N, and every factor ring R/N gives rise to a homomorphism mapping R into R/N."}

\paragraph{Selected Exercises:}

\paragraph{20.} Let $R$ be a commutative ring with unity of prime characteristic $P$. Show that the map $\phi_p: R \to R$ given by $\phi_p(a) = a^p$ is a homomorphism. (the \textit{Frobenius homomorphism})

\begin{flushleft}
\textit{Proof:}\\
$\phi_p(a+b) = (a+b)^p = a^p +b^p$\\
The last step is true because $R$ has a prime characteristic.\\
Therefore, $\phi_p(a+b) = \phi_p(a)+\phi_p(b)$.\\
$\phi_p(ab) = (ab)^p = a^pb^p = \phi_p(a)\phi_p(b)$\\
Therefore, $\phi_p$ is a homomorphism. $\blacksquare$
\end{flushleft}

\paragraph{24.} Show that a factor ring of a field is either the trivial ring of one element or is isomorphic to a field.

\begin{flushleft}
\textit{Proof:} Recall that a field only has 2 possible ideals, $\{0\}$ and $F$, the field itself. Therefore, the possible factor rings are $F/\{0\}$ and $F/F$. $F/\{0\}$ is isomorphic to $F$ and $F/F$ has a single coset that contains everything and is therefore isomorphic to $\{0\}$. $\blacksquare$
\end{flushleft}

\paragraph{30.} An element $a$ of a ring $R$ is \textit{nilpotent} if $a^n=0$ for some $n \in \mathbb{Z}^+$. Show that the collection of all nilpotent elements in a commutative ring $R$ is an ideal, the \textit{nilradical} of $R$.

\begin{flushleft}
\textit{Proof:} Let $a$ be nilpotent and let $r \in R$.\\
$(ar)^n = (ra)^n= r^n \dot a^n = 0$\\
Therefore, $a \cdot r$, $r c\dot a$ are nilpotent.\\
(also need to show that the nilpotents are closed, but that takes a while)
\end{flushleft}








\end{document}